%% BioMed_Central_Tex_Template_v1.06
%%                                      %
%  bmc_article.tex            ver: 1.06 %
%                                       %

%%IMPORTANT: do not delete the first line of this template
%%It must be present to enable the BMC Submission system to
%%recognise this template!!

%%%%%%%%%%%%%%%%%%%%%%%%%%%%%%%%%%%%%%%%%
%%                                     %%
%%  LaTeX template for BioMed Central  %%
%%     journal article submissions     %%
%%                                     %%
%%          <8 June 2012>              %%
%%                                     %%
%%                                     %%
%%%%%%%%%%%%%%%%%%%%%%%%%%%%%%%%%%%%%%%%%


%%%%%%%%%%%%%%%%%%%%%%%%%%%%%%%%%%%%%%%%%%%%%%%%%%%%%%%%%%%%%%%%%%%%%
%%                                                                 %%
%% For instructions on how to fill out this Tex template           %%
%% document please refer to Readme.html and the instructions for   %%
%% authors page on the biomed central website                      %%
%% http://www.biomedcentral.com/info/authors/                      %%
%%                                                                 %%
%% Please do not use \input{...} to include other tex files.       %%
%% Submit your LaTeX manuscript as one .tex document.              %%
%%                                                                 %%
%% All additional figures and files should be attached             %%
%% separately and not embedded in the \TeX\ document itself.       %%
%%                                                                 %%
%% BioMed Central currently use the MikTex distribution of         %%
%% TeX for Windows) of TeX and LaTeX.  This is available from      %%
%% http://www.miktex.org                                           %%
%%                                                                 %%
%%%%%%%%%%%%%%%%%%%%%%%%%%%%%%%%%%%%%%%%%%%%%%%%%%%%%%%%%%%%%%%%%%%%%

%%% additional documentclass options:
%  [doublespacing]
%  [linenumbers]   - put the line numbers on margins

%%% loading packages, author definitions

\documentclass[twocolumn]{bmcart}% uncomment this for twocolumn layout and comment line below
%\documentclass{bmcart}

%%% Load packages
\usepackage{amsthm,amsmath}
\usepackage{siunitx}
\usepackage{mfirstuc}
%\RequirePackage{natbib}
\usepackage[colorinlistoftodos]{todonotes}
\RequirePackage{hyperref}
\usepackage[utf8]{inputenc} %unicode support
%\usepackage[applemac]{inputenc} %applemac support if unicode package fails
%\usepackage[latin1]{inputenc} %UNIX support if unicode package fails
\usepackage[htt]{hyphenat}

\usepackage{array}
\newcolumntype{L}[1]{>{\raggedright\let\newline\\\arraybackslash\hspace{0pt}}p{#1}}

%%%%%%%%%%%%%%%%%%%%%%%%%%%%%%%%%%%%%%%%%%%%%%%%%
%%                                             %%
%%  If you wish to display your graphics for   %%
%%  your own use using includegraphic or       %%
%%  includegraphics, then comment out the      %%
%%  following two lines of code.               %%
%%  NB: These line *must* be included when     %%
%%  submitting to BMC.                         %%
%%  All figure files must be submitted as      %%
%%  separate graphics through the BMC          %%
%%  submission process, not included in the    %%
%%  submitted article.                         %%
%%                                             %%
%%%%%%%%%%%%%%%%%%%%%%%%%%%%%%%%%%%%%%%%%%%%%%%%%


%\def\includegraphic{}
%\def\includegraphics{}

%%% Put your definitions there:
\startlocaldefs
\endlocaldefs


%%% Begin ...
\begin{document}

%%% Start of article front matter
\begin{frontmatter}

\begin{fmbox}
\dochead{Report from 2015 Brainhack Americas (MX)}

%%%%%%%%%%%%%%%%%%%%%%%%%%%%%%%%%%%%%%%%%%%%%%
%%                                          %%
%% Enter the title of your article here     %%
%%                                          %%
%%%%%%%%%%%%%%%%%%%%%%%%%%%%%%%%%%%%%%%%%%%%%%

\title{Optimized implementations of voxel-wise degree centrality and local
functional connectivity density mapping in AFNI}
\vskip2ex
\projectURL{Project URL: \url{http://github.com/ccraddock/afni}}

\author[
addressref={aff1, aff2},
corref={aff1},
email={ccraddock@nki.rfmh.org}
]{\inits{RCC} \fnm{R. Cameron} \snm{Craddock}}
\author[
addressref={aff2},
%
email={daniel.clark@childmind.org}
]{\inits{DJC} \fnm{Daniel J.} \snm{Clark}}

%%%%%%%%%%%%%%%%%%%%%%%%%%%%%%%%%%%%%%%%%%%%%%
%%                                          %%
%% Enter the authors' addresses here        %%
%%                                          %%
%% Repeat \address commands as much as      %%
%% required.                                %%
%%                                          %%
%%%%%%%%%%%%%%%%%%%%%%%%%%%%%%%%%%%%%%%%%%%%%%

\address[id=aff1]{%
  \orgname{Computational Neuroimaging Lab, Center for Biomedical Imaging and
Neuromodulation, Nathan Kline Institute for Psychiatric Research},
  \city{Orangeburg},
  \street{140 Old Orangeburg Rd},
  \postcode{10962},
  \postcode{New York},
  \cny{USA}
}
\address[id=aff2]{%
  \orgname{Center for the Developing Brain, Child Mind Institute},
  \city{New York},
  \street{445 Park Ave},
  \postcode{10022},
  \postcode{New York},
  \cny{USA}
}

%%%%%%%%%%%%%%%%%%%%%%%%%%%%%%%%%%%%%%%%%%%%%%
%%                                          %%
%% Enter short notes here                   %%
%%                                          %%
%% Short notes will be after addresses      %%
%% on first page.                           %%
%%                                          %%
%%%%%%%%%%%%%%%%%%%%%%%%%%%%%%%%%%%%%%%%%%%%%%

\begin{artnotes}
\end{artnotes}

%\end{fmbox}% comment this for two column layout

%%%%%%%%%%%%%%%%%%%%%%%%%%%%%%%%%%%%%%%%%%%%%%
%%                                          %%
%% The Abstract begins here                 %%
%%                                          %%
%% Please refer to the Instructions for     %%
%% authors on http://www.biomedcentral.com  %%
%% and include the section headings         %%
%% accordingly for your article type.       %%
%%                                          %%
%%%%%%%%%%%%%%%%%%%%%%%%%%%%%%%%%%%%%%%%%%%%%%

%\begin{abstractbox}

%\begin{abstract} % abstract
	
%Blank Abstract

%\end{abstract}



%%%%%%%%%%%%%%%%%%%%%%%%%%%%%%%%%%%%%%%%%%%%%%
%%                                          %%
%% The keywords begin here                  %%
%%                                          %%
%% Put each keyword in separate \kwd{}.     %%
%%                                          %%
%%%%%%%%%%%%%%%%%%%%%%%%%%%%%%%%%%%%%%%%%%%%%%

%\vskip1ex

%\projectURL{\url{http://github.com/ccraddock/afni}}
%\projectURL{http://github.com/ccraddock/afni}

% MSC classifications codes, if any
%\begin{keyword}[class=AMS]
%\kwd[Primary ]{}
%\kwd{}
%\kwd[; secondary ]{}
%\end{keyword}

%\end{abstractbox}
%
\end{fmbox}% uncomment this for twcolumn layout

\end{frontmatter}

%{\sffamily\bfseries\fontsize{10}{12}\selectfont Project URL: \url{http://github.com/ccraddock/afni}}

%%% Import the body from pandoc formatted text
\section{Introduction}\label{introduction}

Degree centrality (DC) \cite{Rubinov2010} and local functional
connectivity density (lFCD) \cite{Tomasi2010} are statistics calculated
from brain connectivity graphs that measure how important a brain region
is to the graph. DC (a.k.a. global functional connectivity density
\cite{Tomasi2010}) is calculated as the number of connections a region
has with the rest of the brain (binary DC), or the sum of weights for
those connections (weighted DC) \cite{Rubinov2010}. lFCD was developed
to be a surrogate measure of DC that is faster to calculate by
restricting its computation to regions that are spatially adjacent
\cite{Tomasi2010}. Although both of these measures are popular for
investigating inter-individual variation in brain connectivity,
efficient neuroimaging tools for computing them are scarce. The goal of
this Brainhack project was to contribute optimized implementations of
these algorithms to the widely used, open source, AFNI software package
\cite{Cox1996}.

\section{Approach}\label{approach}

Tools for calculating DC (\texttt{3dDegreeCentrality}) and lFCD
(\texttt{3dLFCD}) were implemented by modifying the C source code of
AFNI's \texttt{3dAutoTcorrelate} tool. \texttt{3dAutoTcorrelate}
calculates the voxel \(\times\) voxel correlation matrix for a dataset
and includes most of the functionality we require, including support for
OpenMP \cite{Dagum1998} multithreading to improve calculation time, the
ability to restrict the calculation using a user-supplied or
auto-calculated mask, and support for both Pearson's and Spearman
correlation.

\subparagraph{\texorpdfstring{\texttt{3dDegreeCentrality}:}{:}}\label{section}

Calculating DC is straight forward and is quick when a correlation
threshold or is used. In this scenario, each of the
\(.5*N_{vox}*(N_{vox}-1)\) unique correlations are calculated, and if
they exceed a user specified threshold (default threshold = 0.0) the
binary and weighted DC value for each of the voxels involved in the
calculation are incremented. The procedure is more tricky if sparsity
thresholding is used, where the top \(P\%\) of connections are included
in the calculation. This requires that a large number of the connections
be retained and ranked - consuming substantial memory and computation.
We optimize this procedure with a histogram and adaptive thresholding.
If a correlation exceeds threshold it is added to a 50-bin histogram
(array of linked lists). If it is determined that the lowest bin of the
histogram is not needed to meet the sparsity goal, the threshold is
increased by the bin-width and the bin is discarded. Once all of the
correlations have been calculated, the histogram is traversed from high
to low, incorporating connections into binary and weighted DC until a
bin is encountered that would push the number of retained connections
over the desired sparsity. This bin's values are sorted into a 100-bin
histogram that is likewise traversed until the sparsity threshold is met
or exceeded. Sparsity is exceeded when the differences between
correlation values are less than \(1.0/(50*100)\).

\begin{table*}[t!]
\caption{\label{stattable}Comparison of the time and memory required by the C-PAC and AFNI implementations to calculate DC (sparsity and correlation threshold) and lFCD on the first resting state scan of the first scanning session for all 36 participants' data in the IBATRT dataset. Values are averaged across the 36 datasets and presented along with standard deviations in parenthesis.}
\begin{tabular}{l l l l l l l l}
 \hline\noalign{\smallskip}
          &            & \multicolumn{2}{c}{DC $\rho \geq 0.6$} & \multicolumn{2}{c}{DC $0.1\%$ Sparsity} & \multicolumn{2}{c}{lFCD $\rho \geq 0.6$} \\
  Method  & Number of Threads & Time (s)       & Mem (GB)            & Time (s)       & Mem (GB)            & Time (s)       & Mem (GB) \\
    \hline\noalign{\smallskip}
  C-PAC   & 1          & 360 (15)       & 6.2(70)             & 360 (15)       & 6.2(70)             & 360 (15)       & 6.2(70)  \\
  AFNI    & 2          & 360 (15)       & 6.2(70)             & 360 (15)       & 6.2(70)             & 360 (15)       & 6.2(70)  \\
  AFNI    & 4          & 360 (15)       & 6.2(70)             & 360 (15)       & 6.2(70)             & 360 (15)       & 6.2(70)  \\
  AFNI    & 8          & 360 (15)       & 6.2(70)             & 360 (15)       & 6.2(70)             & 360 (15)       & 6.2(70)  \\
  \noalign{\smallskip}\hline
\end{tabular}
\end{table*}

\subparagraph{\texorpdfstring{\texttt{3dLFCD}:}{:}}\label{section-1}

lFCD was calculating using a region growing algorithm in which face-,
side-, and corner-touching voxels are iteratively added to the cluster
if their correlation with the target voxel exceeds a threshold (default
threshold = 0.0). Although lFCD was originally defined as the number of
voxels locally connected to the target, we also included a weighted
version.

\subparagraph{Validation:}\label{validation}

Outputs from the newly developed tools were benchmarked to Python
implementations of these measures from the Configurable Pipeline for the
Analysis of Connectomes (C-PAC) \cite{Craddock2013c} using in the
publically shared
\href{http://fcon_1000.projects.nitrc.org/indi/CoRR/html/ibatrt.html}{Intrinsic Brain Activity Test-Retest (IBATRT) dataset}
from the Consortium for Reliability and Reproduciblity\cite{Zuo2014}.

\section{Results}\label{results}

AFNI tools were developed for calculating lFCD and DC from functional
neuroimaging data and have been submitted for inclusion into AFNI. LFCD
and DC maps from the test dataset (illustrated in Fig. \ref{centfig})
are identical to those calculated using C-PAC but required substantially
less time and memory (see Table \ref{stattable}).

\section{Conclusions}\label{conclusions}

Optimized versions of lFCD and DC achieved 4\(\times\) to 30\(\times\)
decreases in computation time compared to C-PAC's Python implementation
and decreased the memory footprint to less than 1 gigabyte. These
improvements will dramatically increase the size of Connectomes analyses
that can be performed using conventional workstations. Making this
implementation available through AFNI ensures that it will be available
to a wide range of neuroimaging researchers who do not have the
wherewithal to implement these algorithms themselves.

%%%%%%%%%%%%%%%%%%%%%%%%%%%%%%%%%%%%%%%%%%%%%%
%%                                          %%
%% Backmatter begins here                   %%
%%                                          %%
%%%%%%%%%%%%%%%%%%%%%%%%%%%%%%%%%%%%%%%%%%%%%%

\begin{backmatter}

\section*{Availability of Supporting Data}
More information about this project can be found at: \url{http://github.com/ccraddock/afni}. Further data and files supporting this project are hosted in the \emph{GigaScience} repository REFXXX.

\section*{Competing interests}
None

\section*{Author's contributions}
RCC and DJC wrote the software, DJC performed tests, and DJC and RCC
wrote the report.

\section*{Acknowledgements}
The authors would like to thank the organizers and attendees of
Brainhack MX and the developers of AFNI. This project was funded in part
by a Educational Research Grant from Amazon Web Services.

  
  
%%%%%%%%%%%%%%%%%%%%%%%%%%%%%%%%%%%%%%%%%%%%%%%%%%%%%%%%%%%%%
%%                  The Bibliography                       %%
%%                                                         %%
%%  Bmc_mathpys.bst  will be used to                       %%
%%  create a .BBL file for submission.                     %%
%%  After submission of the .TEX file,                     %%
%%  you will be prompted to submit your .BBL file.         %%
%%                                                         %%
%%                                                         %%
%%  Note that the displayed Bibliography will not          %%
%%  necessarily be rendered by Latex exactly as specified  %%
%%  in the online Instructions for Authors.                %%
%%                                                         %%
%%%%%%%%%%%%%%%%%%%%%%%%%%%%%%%%%%%%%%%%%%%%%%%%%%%%%%%%%%%%%

% if your bibliography is in bibtex format, use those commands:
\bibliographystyle{bmc-mathphys} % Style BST file
\bibliography{brainhack-report} % Bibliography file (usually '*.bib' )

\end{backmatter}
\end{document}
